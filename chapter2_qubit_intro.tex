
\documentclass[12pt]{article}
\usepackage{amsmath, amssymb}
\usepackage{geometry}
\geometry{a4paper, margin=1in}
\usepackage{graphicx}
\usepackage{float}
\usepackage{booktabs}

\title{第2章 量子比特(Qubit)基础}
\author{}
\date{}

\begin{document}

\maketitle

\section{什么是 Qubit?}

在经典计算中,比特(bit)是最小的信息单位,其取值为 $0$ 或 $1$。  
而量子计算中的量子比特(Qubit)是受量子力学原理支配的信息单位,其状态是经典 0 与 1 的线性叠加,表示为:

\[
|\psi\rangle = \alpha |0\rangle + \beta |1\rangle
\]

其中:
\begin{itemize}
    \item $\alpha, \beta \in \mathbb{C}$ 为复数幅度(称为概率幅)
    \item 满足归一化条件:$|\alpha|^2 + |\beta|^2 = 1$
\end{itemize}

Qubit 的这一性质使其具备\textbf{叠加态(superposition)},可以在计算过程中并行表示多个状态,远超过经典比特的表达能力。

\section{Qubit 的几何表示:Bloch Sphere}

任意 Qubit 的状态可以表示为三维球面(Bloch Sphere)上的一个点。Bloch 球上的任意纯态具有如下形式:

\[
|\psi\rangle = \cos\left(\frac{\theta}{2}\right) |0\rangle + e^{i\phi} \sin\left(\frac{\theta}{2}\right) |1\rangle
\]

其中:
\begin{itemize}
    \item $\theta \in [0, \pi]$:极角(纬度)
    \item $\phi \in [0, 2\pi]$:方位角(经度)
\end{itemize}

例如:
\[
|+\rangle = \frac{1}{\sqrt{2}}(|0\rangle + |1\rangle)
\]

\section{多 Qubit 系统与纠缠态}

多个 Qubit 可组成张量积系统,其维度呈指数增长:  
$n$ 个 Qubit 对应 $2^n$ 维的复向量空间。

特别地,量子计算中还存在\textbf{纠缠态(entangled state)},即系统整体不能被拆解为各 Qubit 的乘积态。例如:

\[
|\Phi^+\rangle = \frac{1}{\sqrt{2}} (|00\rangle + |11\rangle)
\]

这是最常见的 Bell 态之一,具有如下特性:
\begin{itemize}
    \item 单独测量任意一个 Qubit 是随机的
    \item 但两个 Qubit 的测量结果始终一致(完美相关性)
\end{itemize}

纠缠态无法通过经典手段模拟,是量子计算中超越经典计算能力的关键资源。

\section{Qubit 与经典比特的对比}

\begin{table}[H]
\centering
\begin{tabular}{@{}lll@{}}
\toprule
\textbf{属性} & \textbf{经典比特(bit)} & \textbf{量子比特(qubit)} \\ \midrule
状态空间 & $\{0, 1\}$ & $\alpha|0\rangle + \beta|1\rangle$ \\
表达能力 & 离散 & 连续(复球面) \\
可叠加性 & 无 & 有(Superposition) \\
纠缠性 & 无 & 有(Entanglement) \\
测量 & 不变 & 状态坍缩为 0 或 1,具有概率分布 \\ \bottomrule
\end{tabular}
\caption{Qubit 与经典比特的对比}
\end{table}

\end{document}
